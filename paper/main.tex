\documentclass{article}
\usepackage{arxiv}
\usepackage{wrapfig}
\usepackage{amsmath}
\usepackage[utf8]{inputenc}
\usepackage[english, russian]{babel}
\usepackage[T1]{fontenc}
\usepackage{url}
\usepackage{booktabs}
\usepackage{float}
\usepackage{amsfonts}
\usepackage{nicefrac}
\usepackage{microtype}
\usepackage{lipsum}
\usepackage{amssymb}
\usepackage{graphicx}
\usepackage{natbib}
\usepackage{doi}
\usepackage{ dsfont }

\pagestyle{fancy}
\usepackage{comment}

\title{Выбор прогностических моделей в римановых фазовых пространствах}



\author{ Эйнуллаев Алтай Эльшан оглы\\
	Кафедра интеллектуальных систем\\
	\texttt{einullaev.ae@phystech.edu} \\
	%% examples of more authors
	\And
	Вадим Викторович Стрижов \\
	д. ф-м. н.\\
	\texttt{strijov@forecsys.ru} \\
	%% \AND
	%% Coauthor \\
	%% Affiliation \\
	%% Address \\
	%% \texttt{email} \\
	%% \And
	%% Coauthor \\
	%% Affiliation \\
	%% Address \\
	%% \texttt{email} \\
	%% \And
	%% Coauthor \\
	%% Affiliation \\
	%% Address \\
	%% \texttt{email} \\
}
\date{}

\renewcommand{\shorttitle}{\textit{arXiv} Template}

%%% Add PDF metadata to help others organize their library
%%% Once the PDF is generated, you can check the metadata with
%%% $ pdfinfo template.pdf
\hypersetup{
pdftitle={A template for the arxiv style},
pdfsubject={q-bio.NC, q-bio.QM},
pdfauthor={David S.~Hippocampus, Elias D.~Striatum},
pdfkeywords={First keyword, Second keyword, More},
}

\begin{document}




\maketitle

\begin{abstract}
	Матрицы ковариации многомерных временных рядов лежат в римановом пространстве SPD матриц. Свойства этого пространства используются для решения задач классификации многомерных временных рядов. Проблема с использованием этого же подхода для прогнозирования многомерных временных рядов заключается в том, что в отличие от задач классификации, нет возможности решать задачу в пространстве SPD матриц. Предлагаются прогностические модели, использующие риманову геометрию SPD матриц, и исследуется качество их прогноза в зависимости от свойств прогнозируемых временных рядов.

\end{abstract}


\keywords{SPD матрицы \and Касательное пространство \and tSSA \and Матрица ковариаций}

\section{Введение}

Матрицы ковариаций многомерных временных рядов принадлежат гладкому многообразию SPD матриц. Свойства этого пространства \cite{moakher2005differential} могут быть применены для решения различных задач, связанных с многомерными временными рядами. В частности, были предложены методы классификации EEG сигналов, основанные на римановой геометрии пространства SPD матриц \cite{barachant2010riemannian}, \cite{barachant2011multiclass}, \cite{congedo2017riemannian}. Главная идея состоит в том, чтобы перейти в пространство SPD матриц и использовать риманову метрику для решения задачи. Также был разработан метод, в которых матрицы ковариации отображались в касательное пространство, являющееся евклидовым, и в нем классифицировались с помощью известных методов классификации, например SVM. \cite{barachant2013classification}. 

В отличие от классификации, для прогнозирования многомерных временных рядов не разработаны прогностические модели, использующие риманову геометрию матриц ковариации. В первую очередь, это связано с тем, что нельзя решить задачу в пространстве матриц ковариации, т.к. модель должна прогнозировать исходные временные ряды, а не метку, как в задаче классификации. Однако, можно рассматривать векторное представление матриц ковариации как описание взаимосвязи фазовах траекторий многомерных временных рядов и использовать эту информацию для улучшения качества прогноза. 

\section{Постановка задачи}

Ставится задача прогнозирования многомерного временного ряда $\mathbf{x}_t = [x_{1t}, \ldots, x_{nt}]^{\text{T}}$. Каждой компоненте временному ряду можно поставить в соответствие фазовое пространство векторов задержек размерности $L$. Матрица ковариации для $\mathbf{x}_t$ формально определяется следующим образом:

\begin{equation}
    \Sigma = \text{E} (\mathbf{x}_t - \text{E}(\mathbf{x}_t)(\mathbf{x}_t - \text{E}(\mathbf{x}_t))^{\text{T}}.
\end{equation}

Будем использовать выборочую матрицу ковариации, в качестве оценки матрицы ковариации в каждый момент времени:

\begin{equation}
    \mathbf{P}_t = \dfrac{1}{L - 1}\mathbf{X}_t\mathbf{X}_t^{\text{T}},
\end{equation}

где $\mathbf{X}_t = [\mathbf{x}_{t - L + 1}, \ldots \mathbf{x}_{t}]$. Таким образом, $\mathbf{P}_t$ описывает взаимосвязь между точками фазовых пространств различных компонент временного ряда.

\subsection{Риманова геометрия}

Введем определения и инструменты римановой геометрии, необходимые для описания предлагаемых прогностических моделей. Определенные выборочные матрицы ковариации принадлежат к некоторому многообразию. 

\subsubsection{Обозначения}

Пусть $S(n) = \{\mathbf{S} \in M(n), \mathbf{S}^{\text{T}} = \mathbf{S}\}$ -- пространство всех $n \times n$ симметричных матриц в пространстве квадратных вещественных матриц $M(n)$ и $P(n) = \{\mathbf{P} \in S(n), \mathbf{u}^{\text{T}}\mathbf{P}\mathbf{u} > 0, \forall \mathbf{u} \in \mathds{R}\}$. Для SPD матриц из $P(n)$, матричная экспонента вычисляется с помощью  разложения по собственным значениям матрицы $\mathbf{P}$:

\begin{equation}
    \mathbf{P} = \mathbf{U} \, \text{diag}(\sigma_1, \ldots, \sigma_n) \, \mathbf{U}^\text{T}, 
\end{equation}

где $\sigma_1 > \sigma_2 > \ldots > \sigma_n > 0$ -- собственные числа и $\mathbf{U}$ -- матрица собственных векторов $\mathbf{P}$. Тогда

\begin{equation}
    \exp(\mathbf{P}) = \mathbf{U} \, \text{diag}(\exp(\sigma_1), \ldots, \exp(\sigma_n)) \, \mathbf{U}^\text{T}.
\end{equation}

Обратная операция:

\begin{equation}
    \log(\mathbf{P}) = \mathbf{U} \, \text{diag}(\log(\sigma_1), \ldots, \log(\sigma_n)) \, \mathbf{U}^\text{T}.
\end{equation}

Для матриц $\mathbf{P} \in P(n)$, $\mathbf{S} \in S(n)$ справедливы следующие утверждения: $\log(\mathbf{P}) \in S(n)$, $\exp(\mathbf{S}) \in P(n)$. Будем обозначать $\mathbf{P}^{\frac12}$ такую симметричную матрицу $\mathbf{A}$, что $\mathbf{A}\mathbf{A} = \mathbf{P}$. 

\subsubsection{Метрика в римановом пространстве}

Пространство SPD матриц $P(n)$ является гладким многообразием $\mathcal{M}$. Касательное пространство к $P(n)$ в $\mathbf{P}$ -- векторное пространство $T_{\mathbf{P}}$, лежащее в $S(n)$. Размерности многообразия и касательного пространства равны $m = n(n + 1) / 2$. 

В касательном пространстве определено скалярное произведение:

\begin{equation}
    \langle \mathbf{S}_1, \mathbf{S}_2 \rangle_{\mathbf{P}} = \text{tr}(\mathbf{S}_1\mathbf{P}^{-1}\mathbf{S}_2\mathbf{P}^{-1}).
\end{equation}

Пусть $\mathbf{\Gamma}(t) : [0, 1] \rightarrow P(n)$ дифференцируемая кривая из $\Gamma(0) = \mathbf{P}_1$ в $\Gamma(1) = \mathbf{P}_2$. Длина кривой $\mathbf{\Gamma}(t)$ равна

\begin{equation}
    L(\mathbf{\Gamma}(t)) = \int\limits_{0}^1 \|\dot{\mathbf{\Gamma}}(t)\|_{\mathbf{\Gamma}(t)}dt,
\end{equation}

с нормой, порожденной скалярным произведением, определенным выше. Кривая минимальной длины, соединяющая две точки многообразия называется геодезической линией и риманово расстояние между ними равно ее длине. Таким образом, геодезическое расстояние \cite{moakher2005differential}:

\begin{equation}
    \delta(\mathbf{P}_1, \mathbf{P}_2) = \|\log (\mathbf{P}_1^{-1}\mathbf{P}_2)\|_F = \left(\sum\limits_{i = 1}^n \log ^2 \lambda_i\right)^{1/2},
\end{equation}

где $\lambda_i$, $i = 1, \ldots, n$ -- собственные числа матрицы $\mathbf{P}_1^{-1}\mathbf{P}_2$. 

\subsubsection{Касательное пространство}

Для каждой точки $\mathbf{P} \in P(n)$, можем определить касательное пространство, являющееся множеством касательных векторов в $\mathbf{P}$. Каждый касательный вектор $\mathbf{S}_i$ является производной в $t = 0$ геодезической линии $\mathbf{\Gamma}_i(t)$ между $\mathbf{P}$ и экспоненциальным отображением $\mathbf{P}_i = \text{Exp}_{\mathbf{P}}(\mathbf{S}_i)$, определенным следующим образом:

\begin{equation}
    \mathbf{P}_i = \text{Exp}_{\mathbf{P}}(\mathbf{S}_i) = \mathbf{P}^{\frac12} \exp( \mathbf{P}^{-\frac12}\mathbf{S}_i\mathbf{P}^{-\frac12})\mathbf{P}^{\frac12}.
\end{equation}

Обратное отображение:

\begin{equation}
    \mathbf{S}_i = \text{Log}_{\mathbf{P}}(\mathbf{P}_i) = \mathbf{P}^{\frac12} \log( \mathbf{P}^{-\frac12}\mathbf{P}_i\mathbf{P}^{-\frac12})\mathbf{P}^{\frac12}.
\end{equation}

Таким образом, с помощью логарифмического и экспоненциального отображения можем переходить в касательное пространство и обратно.

\begin{figure}[h]
    \centering
    \includegraphics[width=\linewidth]{images/expmap.png}
  \caption{Касательное пространство в точке $\mathbf{P}$, $\mathbf{S}_i$ -- касательный вектор в $\mathbf{P}$, $\Gamma_i(t)$ -- кратчайшая линия, соединяющая $\mathbf{P}$, $\mathbf{P}_i$ в $P(n)$.}
\end{figure}

\subsection{Прогностическая модель}

Требуется выбрать две модели $f$, $h$:

\begin{equation}
    \mathbf{x}_t \xrightarrow{f} \hat{\mathbf{x}}_{t+1},
\end{equation}

\begin{equation}
    \mathbf{x}_t \xrightarrow{g} \mathbf{S}_{t} \xrightarrow{h} \hat{\mathbf{S}}_{t + 1},
\end{equation}

где $f$ -- модель прогнозирования временного ряда в фазовом пространстве, $g$ -- отображение, ставящее в соответствие временному ряду представление матрицы ковариации в касательном пространстве $\mathbf{S}_t$, $h$ -- модель прогнозирования $\hat{\mathbf{S}}_{t + 1}$ в касательном пространстве. Для улучшение прогноза $f$ строится мультимодель:

\begin{equation}
    \hat{\mathbf{x}}_{t + 1} = F(\hat{\mathbf{x}}_{t + 1}, \hat{\mathbf{S}}_{t + 1}) = F(f(\mathbf{x}_t), h(g(\mathbf{x}_t))).
\end{equation}

\section{Вычислительный эксперимент}

\section{Сравнение результатов}

\section{Заключение}


\bibliographystyle{unsrt}
\bibliography{biblio}

\end{document}
